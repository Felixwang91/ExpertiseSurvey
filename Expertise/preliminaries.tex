\thesistitle{Expertise Location Approaches and Systems in Software Engineering}

%"Dissertation" for PhD, "Thesis" for master's
\documenttitle{Thesis}

\degreename{Master of Science}

% Use the wording given in the official list of degrees awarded by UCI:
% http://www.rgs.uci.edu/grad/academic/degrees_offered.htm
\degreefield{Software Engineering}

% Your name as it appears on official UCI records.
\authorname{Zhendong Wang}

% Use the full name of each committee member.
\committeechair{David Redmiles}
\othercommitteemembers
{
  James Jones\\
  Yi Wang
}

\degreeyear{2018}

\copyrightdeclaration
{
  {\copyright} {\Degreeyear} \Authorname
}

% If you have previously published parts of your manuscript, you must list the
% copyright holders; see Section 3.2 of the UCI Thesis and Dissertation Manual.
% Otherwise, this section may be omitted.
% \prepublishedcopyrightdeclaration
% {
% 	Chapter 4 {\copyright} 2003 Springer-Verlag \\
% 	Portion of Chapter 5 {\copyright} 1999 John Wiley \& Sons, Inc. \\
% 	All other materials {\copyright} {\Degreeyear} \Authorname
% }

% The dedication page is optional
% (comment out to exclude).
% \dedications
% {
%   (Optional dedication page)
  
%   To ...
% }

\acknowledgments
{
  I would like to thank...
  
  (You must acknowledge grants and other funding assistance. 
  
  You may also acknowledge the contributions of professors and
  friends.
  
  You also need to acknowledge any publishers of your previous
  work who have given you permission to incorporate that work
  into your dissertation. See Section 3.2 of the UCI Thesis and
  Dissertation Manual.)
}


% Some custom commands for your list of publications and software.
\newcommand{\mypubentry}[3]{
  \begin{tabular*}{1\textwidth}{@{\extracolsep{\fill}}p{4.5in}r}
    \textbf{#1} & \textbf{#2} \\ 
    \multicolumn{2}{@{\extracolsep{\fill}}p{.95\textwidth}}{#3}\vspace{6pt} \\
  \end{tabular*}
}
\newcommand{\mysoftentry}[3]{
  \begin{tabular*}{1\textwidth}{@{\extracolsep{\fill}}lr}
    \textbf{#1} & \url{#2} \\
    \multicolumn{2}{@{\extracolsep{\fill}}p{.95\textwidth}}
    {\emph{#3}}\vspace{-6pt} \\
  \end{tabular*}
}

% Include, at minimum, a listing of your degrees and educational
% achievements with dates and the school where the degrees were
% earned. This should include the degree currently being
% attained. Other than that it's mostly up to you what to include here
% and how to format it, below is just an example.
%
% CV is required for PhD theses, but not Master's
% comment out to exclude
% \curriculumvitae
% {

% \textbf{EDUCATION}
  
%   \begin{tabular*}{1\textwidth}{@{\extracolsep{\fill}}lr}
%     \textbf{Doctor of Philosophy in Computer Science} & \textbf{2012} \\
%     \vspace{6pt}
%     University name & \emph{City, State} \\
%     \textbf{Bachelor of Science in Computational Sciences} & \textbf{2007} \\
%     \vspace{6pt}
%     Another university name & \emph{City, State} \\
%   \end{tabular*}

% \vspace{12pt}
% \textbf{RESEARCH EXPERIENCE}

%   \begin{tabular*}{1\textwidth}{@{\extracolsep{\fill}}lr}
%     \textbf{Graduate Research Assistant} & \textbf{2007--2012} \\
%     \vspace{6pt}
%     University of California, Irvine & \emph{Irvine, California} \\
%   \end{tabular*}

% \vspace{12pt}
% \textbf{TEACHING EXPERIENCE}

%   \begin{tabular*}{1\textwidth}{@{\extracolsep{\fill}}lr}
%     \textbf{Teaching Assistant} & \textbf{2009--2010} \\
%     \vspace{6pt}
%     University name & \emph{City, State} \\
%   \end{tabular*}

% \pagebreak

% \textbf{REFEREED JOURNAL PUBLICATIONS}

%   \mypubentry{Ground-breaking article}{2012}{Journal name}

% \vspace{12pt}
% \textbf{REFEREED CONFERENCE PUBLICATIONS}

%   \mypubentry{Awesome paper}{Jun 2011}{Conference name}
%   \mypubentry{Another awesome paper}{Aug 2012}{Conference name}

% \vspace{12pt}
% \textbf{SOFTWARE}

%   \mysoftentry{Magical tool}{http://your.url.here/}
%   {C++ algorithm that solves TSP in polynomial time.}

% }

% The abstract should not be over 350 words, although that's
% supposedly somewhat of a soft constraint.
\thesisabstract
{
Successful software engineering activities require qualified software developers with proper expertise. Although expertise has been studied for many years and various expertise location approaches have been postulated, new approaches and opportunities are emerging today because of the rise of code hosting and knowledge sharing sites. As a step towards understanding the past work and the present opportunities in the context of today's software engineering practice, we perform a systematic literature survey. In analyzing the literature, we identify two broad categories of expertise research: 1) identifying the characteristics of experts, and 2) locating experts. The studies in the latter category can be further classified into three subcategories, which are: i) locating expertise by leveraging the organizational setting; ii) by mining historical artifacts; and iii) by knowledge sharing. Our analysis also identifies the major limitations of existing work, including a disconnection between early expertise studies and current location approaches; an over reliance on the experience-based model to measure expertise; a neglect on the constraints for coordination; and a lack of empirical evaluation in the real-world context, among others. Finally, we highlight research trends and promising directions for future research.
}


%%% Local Variables: ***
%%% mode: latex ***
%%% TeX-master: "thesis.tex" ***
%%% End: ***
